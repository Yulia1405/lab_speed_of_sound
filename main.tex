\documentclass{article}
\usepackage[utf8]{inputenc}

\title{Lab: Measurement of the speed of sound}
\author{Yulia Dichenko }
\date{February 2016}

\begin{document}

\maketitle

\section{Question:}

\text{What is the velocity of sound in the chamber, using forks?}

\section{Definitions:}

\text{Wave speed (v) - the speed at which a wave travels. Wave speed is related to wavelength, frequency, and period. The most commonly used wave speed is the speed of visible light, an electromagnetic wave.

Frequency (f) - the number of crests of a wave that move past a given point in a given unit of time. The most common unit of frequency is the hertz (Hz), corresponding to one crest per second.

Wavelength ($\lambda$) -  the distance between consecutive corresponding points of the same phase, such as crests, troughs, or zero crossings and is a characteristic of both traveling waves and standing waves, as well as other spatial wave patterns.

Standing wave - also known as a stationary wave – is a wave in a medium in which each point on the axis of the wave has an associated constant amplitude. The locations at which the amplitude is minimum are called nodes, and the locations where the amplitude is maximum are called anti-nodes.

Resonance - is a phenomenon that occurs when a vibrating system or external force drives another system to oscillate with greater amplitude at a specific preferential frequency.}

\section{Materials:}

\text{Chamber

Water

Fork}

\section{Method:}

\text{First, we calculated predicted $\lambda$ for each of the forks by using formula $\lambda=\frac{V}{f}$

Second, we put water, which would be on the predicted distance from a fork.

Third, we measured and understood on which distance from fork the sound is the loudest, which means the wave reaches its peak.}

\section{Data:}

\text{1st fork $f$ - 512 Hz

2nd fork $f$ - 384 Hz

3rd fork $f$ - 256 Hz

$V$ for all - 340 $\frac{\text{m}}{\text{s}}$ }

\section{Theory:}

\text{1. $f$= 512 Hz} $\lambda=\frac{V}{f}=\frac{340\frac{m}{s}}{512 Hz}=0.664 m$

$L=\frac{\lambda}{4}=\frac{0.664 cm}{4}=0.166m = 16.6cm$

\text{2. $f$= 384 Hz}

$\lambda=\frac{V}{f}=\frac{340\frac{m}{s}}{384 Hz}=0.888 m$

$L=\frac{\lambda}{4}=\frac{0.888 cm}{4}=0.222m = 22.2cm$

\text{3. $f$= 256 Hz}

$\lambda=\frac{V}{f}=\frac{340\frac{m}{s}}{256 Hz}=1.332 m$

$L=\frac{\lambda}{4}=\frac{1.332 cm}{4}=0.333m = 33.3cm$

\section{Results:}

\text{1. L = 16.5 cm}

$V=4*L*f=4*0.165*512=338\frac{m}{s}$

\text{2. L = 21.5 cm}

$V=4*L*f=4*0.215*384=330\frac{m}{s}$

\text{3. L = 32.0 cm}

$V=4*L*f=4*0.32*256=327\frac{m}{s}$

\section{Discussion of an error:}

\text{Error can take place in this experiment in various places. It could be not exact measurement of a water in a chamber, because the measurement were made by eyes.

Also the error can be because of the leap on top of the chamber, which changes the wave.

Forks are a little bit bend, so the frequency could be not exact.}

\section{Conclusion:}

\text{There are different possible ways of measuring speed of sound. We could have used the formula of $D=V*t$, which represent sound as a moving object, but it is wrong, because we need to understand the nature of the wave and the behaviour of wave. As I understood it is very important to count error in this experiment.}

\end{document}
